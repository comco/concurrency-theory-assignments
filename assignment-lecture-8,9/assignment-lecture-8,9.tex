\documentclass[a4paper]{article}

\usepackage[utf8]{inputenc}
\usepackage[english]{babel}

\usepackage{amsmath,amsthm,amssymb}
\usepackage{geometry} % for correct margins
\usepackage{graphicx}
\usepackage{listings}
\usepackage{hyperref}

\newcommand{\prog}[1]{\texttt{#1}}
\newcommand{\pfun}[1]{\textsf{#1}}

\lstset{basicstyle=\ttfamily,
language=C,
}

\title{Concurrency Theory, Assignment Lecture 8,9}
\author{Krasimir Georgiev}

\begin{document}
\maketitle

The following PSF specification \texttt{Factory.psf} correctly models the
manufacturing process.

It is possible to arrive at the output sequence \texttt{output(C) . output(B) .
    output(A)}, starting from the input sequence \texttt{input(A) . input(B) .
input(C)}, as demonstrated by the following trace:
\lstinputlisting{trace.txt}

The source code of this assignment can be found on
\href{https://github.com/comco/concurrency-theory-assignments/tree/master/assignment-lecture-8,9}{GitHub}.
\end{document}
